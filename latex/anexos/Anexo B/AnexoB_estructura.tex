%preambulo
\documentclass[12pt, a4paper]{report} 

\usepackage[utf8]{inputenc} 
\usepackage[T1]{fontenc} 
\usepackage[spanish]{babel} 
\usepackage{amsmath, amsfonts, amssymb} 
\usepackage{fancyhdr} 
\usepackage{titlesec}
\usepackage{graphicx} 
\usepackage{booktabs} 
\usepackage{float}
\usepackage{caption}
\captionsetup[table]{labelformat=simple} % Aqui para la nomenclatura de las tablas
%Plots / señales... 
\usepackage{pgfplots}
\pgfplotsset{compat=1.18}
\usepackage{subcaption} %Subfiguras

%Margenes
\usepackage[a4paper,
            top=2.54cm,
            bottom=2.54cm,
            left=2.54cm,
            right=2.54cm]{geometry}

%Links 
\usepackage{hyperref} 
\hypersetup{
    colorlinks=false,
    }
\hypersetup{
  colorlinks=true,
  linkcolor=black,      % enlaces internos (como footnotes)
  citecolor=black,      % enlaces a bibliografía
  urlcolor=blue,        % enlaces web
  pdfborder={0 0 0},     % <-- elimina los recuadros de 
  pdftitle={TFG ACI},
  pdfpagemode=FullScreen
}
%Codigo
\usepackage{listings} 
\usepackage{minted}
\usepackage{xcolor}
\usepackage[final]{pdfpages}
\usepackage{lipsum}
\usepackage{changepage}
\usepackage{xtab}
\usepackage[most]{tcolorbox}
\usepackage[shortlabels]{enumitem}
%Referencias 
\usepackage{csquotes}

\usepackage[
    backend=biber,
    sorting=ynt,
    style=apa
]{biblatex}
\addbibresource{referencias.bib}
\fancyhf{} % Limpia encabezado
% Capitulos       
\titleformat{\chapter}[display]
  {\normalfont\huge\bfseries}
  {\chaptername\ \thechapter}{10pt}{\Huge}
\titlespacing*{\chapter}{0pt}{-60pt}{30pt}
%Bibliografía 
\addbibresource{referencias.bib}

\begin{document}

\chapter*{Boceto Anexo B}
\section*{1. Introducción}
La ingeniería de telecomunicaciones desempeña un papel clave en el desarrollo de tecnologías basadas en la electrónica, la instrumentación y el procesamiento avanzado de señales. Uno de los campos emergentes donde estas disciplinas convergen es el ámbito de la ingeniería biomédica, concretamente en el análisis y procesado de señales fisiológicas para aplicaciones de rehabilitación y control de prótesis. (A expandir)
\section*{2. Estado de la cuestión}
Actualmente se dispone de una gran variedad de estudios que investigan los algoritmos de separación de señales y sus aplicaciones en ámbitos como imágenes, sonido e incluso señales EEG. No obstante, la implementación de estas técnicas en señales electromiográficas permanece relativamente poco explorada, a pesar de su potencial para ofrecer información valiosa sobre músculos reinervados.  (A expandir)
\section*{3. Motivación}
Cuando una persona pierde una extremidad, los nervios motores de la terminación quedan seccionados y, a menos que se reconecten, terminan formando "nervios dolorosos" (terminaciones nerviosas desorganizadas). Por ello, con el objetivo de mejorar la integración neuromuscular, se redirigen los nervios motores a músculos intactos que ya no tienen utilidad con el fin de que estos se conviertan en amplificadores biológicos para las señales. Este proceso se conoce como reinnervación muscular y es vital para la mejora del control en prótesis. No obstante, al hacer uso de músculos auxiliares como bioamplificadores, estos interaccionan con el nervio redirigido y modifican la forma y características de la señal emitida. (A expandir)

Este trabajo Fin de Grado busca diseñar e implementar un algoritmo que sea capaz de identificar la señal del nervio, procesarla y separarla de la interacción del músculo auxiliar. De esta manera, se busca mejorar la precisión y robustez en el control de prótesis basadas en EMG.

\section*{4. Objetivos del proyecto}
A continuación se detallan los objetivos del Trabajo Fin de Grado, explicándose más a fondo en la versión final del Anexo B.  
\begin{itemize}
    \item Analizar y tratar la señal del músculo reinnervado.
    \item Separación de la señal EMG, resultando en la señal del nervio y del músculo auxiliar
    \item Discusión de varios algoritmos, exponiendo sus pros y contras, y finalmente seleccionando uno para su implementación.
    \item Probar eficacia en una base real de señales EMG.
    \item Explicar el uso y comportamiento del algortimo para el público menos técnico (ejemplo: manual de uso para profesionales sanitarios).
    
\end{itemize}

\section*{5. Alineación con ODS}
\href{Link}{https://sdgs.un.org/goals}
\begin{itemize}
    \item Objetivo 3º: Salud y bienestar
    \item Objetivo 9º: Industria, innovación e infraestructura.
\item Objetivo 16º: Reducción de desigualdades
\end{itemize}
En la versión final se explicará el papel que desempeña este TFG en cada uno de los objetivos mencionados.
\section*{6. Metodología del trabajo}
En esta sección se expondrá la planificación del trabajo para su adecuada finalización. Se nombrarán los diferentes algoritmos de los que se dispone, pruebas que se realizarán y los procedimientos que sean pertinentes. Tras su explicación en detalle, se ilustrará con un cronograma o un diagrama de Gantt.
\\La estrategia a seguir para este proyecto se basa en los siguientes pasos:
\begin{itemize}
    \item Preprocesamiento
    \item Separación de señales
    \begin{itemize}
    \item ICA (Separación a ciegas pura)
    \begin{itemize}
        \item FastICA
        \item JADE
        \item Infomax
    \end{itemize}
    \item Separación con referencia (ex: pre-TMR, musculo auxiliar ...)
    \begin{itemize}
        \item Kernel Constrained ICA
        \item Correlation Constrained ICA
        \item 2CFastICA (Kernel and Constrained)
    \end{itemize} 
\end{itemize}
    \item Identificación del componente 
    \item Análisis y validación
\end{itemize}


Se detallará más a fondo su lógica y funcionamiento en la versión final del anexo B.
\section*{7. Recursos empleados}
fLos recursos a emplear en este trabajo incluyen otros algoritmos existentes, así como lenguajes y herramientas de programación. Entre ellos se encuentran python, matlab, Github, Zotero...

A continuación se adjuntan referencias que posiblemente se emplearán a lo largo del proyecto:
\cite{todd_kuiken_electromyography_2023, tang_improving_2025, naik_blind_2014, chen_2cfastica_2024, huang_performance_2025}


\printbibliography

\end{document}