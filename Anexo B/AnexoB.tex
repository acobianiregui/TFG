\documentclass[12pt, a4paper]{article} 

\usepackage[utf8]{inputenc} 
\usepackage[T1]{fontenc} 
\usepackage[spanish]{babel} 
\usepackage{amsmath, amsfonts, amssymb} 
\usepackage{fancyhdr} 
\usepackage{titlesec}
\usepackage{graphicx} 
\usepackage{booktabs} 
\usepackage{float}
\usepackage{ulem}
\usepackage{caption}

\captionsetup[table]{labelformat=simple} % Aqui para la nomenclatura de las tablas
%Plots / señales... 
\usepackage{pgfplots}
\pgfplotsset{compat=1.18}
\usepackage{subcaption} %Subfiguras

%Margenes
\usepackage[a4paper,
            top=2.54cm,
            bottom=2.54cm,
            left=2.54cm,
            right=2.54cm]{geometry}

%Links 
\usepackage{hyperref} 
\hypersetup{
    colorlinks=false,
    }
\hypersetup{
  colorlinks=true,
  linkcolor=black,      % enlaces internos (como footnotes)
  citecolor=black,      % enlaces a bibliografía
  urlcolor=blue,        % enlaces web
  pdfborder={0 0 0},     % <-- elimina los recuadros de 
  pdftitle={TFG ACI},
  pdfpagemode=FullScreen
}
%Codigo
\usepackage{listings} 
\usepackage{minted}
\usepackage{xcolor}
\usepackage[final]{pdfpages}
\usepackage{lipsum}
\usepackage{changepage}
\usepackage{xtab}
\usepackage[most]{tcolorbox}
\usepackage[shortlabels]{enumitem}
\usepackage{matlab-prettifier} %Codigo matlab
%Referencias 
\usepackage{csquotes}

\usepackage[
    backend=biber,
    sorting=ynt,
    style=apa
]{biblatex}
\addbibresource{referencias.bib}
\fancyhf{} % Limpia encabezado

% Cabecera 
\fancyhead[L]{\includegraphics[height=25pt]{imagenes/comillas_icai_logo.jpeg}}          
\fancyhead[C]{Anexo B}        
\fancyhead[R]{A.C.I}
\setlength{\headsep}{50pt}  %Modificar salto entre cabecera y contenido


% Pie de pagina 
\fancyfoot[C]{\thepage}  

% Capitulos       
\titleformat{\chapter}[display]
  {\normalfont\huge\bfseries}
  {\chaptername\ \thechapter}{10pt}{\Huge}
\titlespacing*{\chapter}{0pt}{-70pt}{30pt}

\makeatletter
\def\@makechapterhead#1{%
  {\parindent \z@ \raggedright \normalfont
    \ifnum \c@secnumdepth >\m@ne
        \huge\bfseries \thechapter\quad #1\par\nobreak
        \vskip 40pt
    \fi}}
\makeatother


\usepackage{listings}
\usepackage{xcolor}
\let\chapter\section
\let\chaptername\sectionname

\begin{document}
\begin{titlepage}
	\begin{figure}
	\centering
	\includegraphics[width=0.6\linewidth]{imagenes/imagenes-web-8.jpeg}
	\end{figure}
	\vspace*{1.5em} 
	\centering
	\Large ESCUELA TÉCNICA SUPERIOR DE INGENIERÍA (ICAI) \\ % Modify accordingly

	\vspace*{2.5em}
	\centering
	\Large Grado en Ingeniería en Tecnologías de Telecomunicación \\ % Modify accordingly
	\vspace*{1em}
    \Large Trabajo Fin de Grado \\ % Modify accordingly
	\vspace*{1em}
	\textbf{Anexo B} % Write here the title of your study
	\\ \large
	\vspace*{3em}
	\begin{minipage}[t]{0.45\textwidth}
        \centering
        Autor \\
        \vspace{0.5em}
        Antón Cobián Iregui \\
        \vspace{2em}
        \includegraphics[width=0.7\linewidth]{imagenes/firma-anton.png} % Tu firma escaneada
    \end{minipage}
    \hfill
    \begin{minipage}[t]{0.45\textwidth}
        \centering
        Dirigido por \\
        \vspace{0.5em}
        Romano Giannetti \\
        \vspace{2em}
        %\includegraphics[width=0.7\linewidth]{imagenes/firma-director.png} % Firma del director (opcional)
    \end{minipage}
	% {\vfill \vspace*{3em}{Author's signature:\hrulefill \hfill} \hfill \\ \vspace*{4em}Supervisors' signatures:\hrulefill \hfill} % Uncoment this line if you would like to add a signature space
	\vfill
    \vspace{2cm}
    \
	Madrid \\
	Mayo 2026 
\end{titlepage}



\pagestyle{fancy}

\section{Introducción}

La ingeniería de telecomunicaciones desempeña un papel clave en el desarrollo de tecnologías basadas en la electrónica, la instrumentación y el procesamiento avanzado de señales. Uno de los campos emergentes donde estas disciplinas convergen es el ámbito de la ingeniería biomédica, concretamente en el análisis y procesamiento de señales fisiológicas para aplicaciones de rehabilitación y control de prótesis. 
%Hablar de la separación de señales en teleco

Específicamente, el estudio de señales como el electromiograma (EMG) permite obtener información detallada sobre la actividad muscular, fundamental para comprender el funcionamiento del sistema neuromuscular, diseñar sistemas de asistencia y optimizar los procesos de rehabilitación. La correcta interpretación de estas señales depende tanto de la adquisición precisa de los datos como del empleo de técnicas avanzadas de procesamiento, incluyendo extracción de características y algoritmos de separación de fuentes.

Adicionalmente, la separación de señales es una herramienta comúnmente aplicada en el mundo de las telecomunicaciones dada la creciente importancia de comunicaciones inalámbricas y radios inteligentes. En estas tecnologías, se dan ocasiones donde múltiples transmisores emiten señales en frecuencias cercanas que se solapan entre sí y otras interferencias y, en consecuencia, surge una necesidad de separar estas señales para poder recuperar la información deseada y monitorizar el espectro. No obstante, todas estas aplicaciones, a pesar de ser puramente de telecomunicaciones, contienen ideas valiosas que pueden ser aplicadas a señales biomédicas como EMG.

Con todo, este Trabajo Fin de Grado pretende centrarse en el desarrollo de algoritmos de separación para señales EMG, con el objetivo de mejorar la calidad de las señales procesadas y facilitar su uso en entornos clínicos y de investigación. 


\chapter{Estado de la cuestión}

Artículos científicos publicados durante los últimos años han demostrado que los métodos de separación ciega de de fuentes (BSS), en concreto algoritmos basados en el Análisis de Componentes Independientes (ICA), ofrecen resultados prometedores en la separación de señales ocultas o mezcladas. Por lo general, su uso y validación es frecuente en campos como el procesamiento de audio, imágenes o señales de electronencefalograma (EEG). Concretamente en EEG es crucial emplear algún tipo de separación dada la extensa mezcla de señales que se produce en el cerebro. 
%IMAGENES DE EEG
\begin{figure}[H]
    \centering
    \begin{subfigure}[b]{0.45\textwidth}
        \centering
        \includegraphics[width=\textwidth]{Anexo B/figuras/EEG_preICA.png}
        \caption{Canales EEG pre-ICA}
        \label{fig:1a}
    \end{subfigure}
    \hfill
    \begin{subfigure}[b]{0.5\textwidth}
        \centering
        \includegraphics[height=0.15\textheight]{Anexo B/figuras/EEG_postICA.png}
        \caption{Separaciones resultantes post-ICA}
        \label{fig:1b}
    \end{subfigure}
    \caption{Aplicación de ICA en EEG, detallada por \textbf{\textit{\cite{naik_blind_2014}}}}
    \label{fig:comparacion}
\end{figure}


En este contexto, diversos estudios, como el publicado por \textbf{\textit{\cite{chen_2cfastica_2024}}}, han explorado la viabilidad de utilizar ICA para descomponer señales EMG multicanal y aislar componentes asociados a músculos individuales o grupos de unidades motoras. Estos trabajos muestran que la estructura de mezcla lineal de la sEMG, junto con la independencia relativa entre fuentes fisiológicas, convierte a ICA en un candidato adecuado para mejorar la interpretabilidad de las grabaciones. 

No obstante, en poblaciones con reinervación, la independencia entre fuentes puede verse reducida debido a la reorganización de las conexiones motoras, lo que plantea nuevos retos metodológicos y explica la ausencia de un consenso claro.

A pesar de estas limitaciones, el interés por el análisis avanzado de EMG ha aumentado de forma notable en los últimos años. Este crecimiento se observa especialmente en áreas como la rehabilitación neuromuscular, la monitorización de la reinervación tras lesiones periféricas, el control de prótesis y el estudio de trastornos del movimiento. Como consecuencia, han surgido numerosas investigaciones que evalúan distintas variantes de ICA y métodos relacionados, entre ellos FastICA, JADE o Infomax. Los resultados obtenidos muestran que la separación de fuentes puede complementar a las técnicas tradicionales, ofreciendo una caracterización más precisa de la actividad muscular y facilitando la identificación de patrones asociados a procesos de reinervación.

En resumen, el estado actual de la investigación revela un campo , donde el uso de separación de señales EMG se encuentra en una fase emergente pero prometedora. Los estudios actuales podrían indicar que la separación de fuentes podría convertirse en una clave para mejorar evaluación clínica de músculos reinvervados y optimizar dispositivos médicos como prótesis.
 
\clearpage
\chapter{Motivación}
Cuando una persona pierde una extremidad, los nervios motores de la terminación quedan seccionados y, a menos que se reconecten, terminan formando "nervios dolorosos" (terminaciones nerviosas desorganizadas). Por ello, con el objetivo de mejorar la integración neuromuscular, se redirigen los nervios motores a músculos intactos que ya no tienen utilidad con el fin de que estos se conviertan en amplificadores biológicos para las señales. Este proceso se conoce como reinnervación muscular o \textbf{\textit{Targeted Muscle Reinnervation (TMR)}} y es vital para la mejora del control en prótesis.
\begin{figure}[H]
    \centering
    \begin{subfigure}[b]{0.4\textwidth}
        \centering
        \includegraphics[width=\textwidth, height=0.93\textwidth]{Anexo B/figuras/TMR_anterior.png}
        \caption{Visión anterior del esquema}
        \label{fig:1a}
    \end{subfigure}
    \hfill
    \begin{subfigure}[b]{0.45\textwidth}
        \centering
        \includegraphics[width=\textwidth]{Anexo B/figuras/TMR_posterior.png}
        \caption{Visión posterior del esquema}
        \label{fig:1b}
    \end{subfigure}
    \caption{Esquema del plan quirúrgico para TMR en amputación transhumeral, representada por \textbf{\textit{\cite{cheesborough_targeted_2015}}}. Etiquetas del mismo color representan los músculos y su origen de inervación.}
    \label{fig:comparacion}
\end{figure}

Sin embargo, al hacer uso de músculos auxiliares como bioamplificadores, estos interaccionan con el nervio redirigido y modifican la forma y características de la señal emitida. Esto se debe a que el desvío del nervio introduce cambios en la geometría y densidad de las fuentes de las unidades motoras, afectando a la transferencia del potencial a través de tejidos y, en consecuencia, la forma temporal y espacial del potencial de acción \textit{(MUAP)}. Resumiendo, la señal EMG captada por los electrodos puede presentar retardos, variaciones en la amplitud y patrones de activaciones atípicas.


Por ello, surge la necesidad de este trabajo Fin de Grado, buscando diseñar e implementar un algoritmo que sea capaz de identificar la señal del nervio, procesarla y separarla de la interacción del músculo auxiliar. De esta manera, se busca mejorar la precisión y robustez en el control de prótesis basadas en EMG.
\clearpage
\chapter{Objetivos del proyecto}
Para realizar este Trabajo Fin de Grado de manera satisfactoria, es crucial establecer las metas que guían el desarrollo del mismo. Por ello, en esta sección se definirán los objetivos que orientan tanto la parte teórica como la práctica del proyecto para así asegurar una estructura coherente y enfoque preciso.

A continuación, se presentan los objetivos generales y específicos que guiarán el desarrollo del proyecto.

\begin{itemize}
    \item Analizar y tratar la señal del músculo reinnervado.
 \end{itemize}   
    %Explicacion aqui
    \noindent Cómo se mencionó en la sección introductoria de este documento, las señales de los músculos reinnervados se ven alterada por la interacción con el músculo auxiliar y, por lo tanto, sus características son considerablemente diferentes respecto a la señal de un músculo sano. Comprender las diferencias entre ambas señales en cuanto amplitud, frecuencia, fase será un aspecto clave para poder completar el trabajo.
\begin{itemize}
    \item Separación de la señal EMG, resultando en la señal del nervio y del músculo auxiliar
\end{itemize}
\noindent Una vez analizada la señal del músculo reinnervado e identificadas sus características, se puede emplear esta información para diseñar y entrenar un modelo que separe la señal . En otras palabras, se busca 
\begin{itemize}
    \item Discusión de varios algoritmos, exponiendo sus pros y contras, y finalmente seleccionando uno para su implementación.
\end{itemize}
\begin{itemize}
    \item Probar eficacia en una base real de señales EMG.
\end{itemize}
    \noindent Por lo general, los algoritmos de separación de fuentes consiguen señales que maximicen su independencia entre si, pero en muchos casos resulta imposible recuperar las fuente originales.  Por ello, probablemente el objetivo más importante de este proyecto es evaluar la actuación del algoritmo desarrollado y considerar si sus resultados pueden conllevar una mejora sustancial en el manejo de prótesis.
\begin{itemize}
    \item Explicar el uso y comportamiento del algortimo para el público menos técnico (ejemplo: manual de uso para profesionales sanitarios).
\end{itemize}
\noindent Los conceptos que se abordan en este Trabajo Fin de Grado, como técnicas de aprendizaje automático (\textit{Machine Learning}), estadística avanzada y procesado de señal, requieren una sólida base en ingeniería. No obstante, en el ámbito clínico cada vez es más común la utilización de herramientas tecnológicas basadas en estos principios, especialmente en áreas como la rehabilitación y el control neuromuscular mediante EMG. Por ello, en el trabajo explorará la necesidad de traducir deel funcionamiento del algoritmo a un lenguaje claro y práctico, facilitando su integración en entornos clínicios y garantizando un uso adecuado y seguro por parte de los especialistas.
\clearpage
\chapter{Alineación con Objetivos de Desarrollo Sostenible}
El análisis y separación de señales electromiográficas (EMG) que se desarrolla a lo largo de este Trabajo Fin de Grado no solo busca aportar avances en el ámbito del procesamiento de señales biomédicas, sino que también se integra de forma natural en varios Objetivos de Desarrollo Sostenible (también conocidos como \href{https://sdgs.un.org/goals}{ODS}). La aplicación de técnicas como preprocesamiento, la eliminación de artefactos y métodos de separación y clasificación contribuye al desarrollo de soluciones tecnológicas que mejoran la salud, impulsan la innovación y promueven la igualdad en el acceso a herramientas biomédicas. 

Por ello, a continuación se mencionan los principales ODS que se persiguen y su papel en el Trabajo Fin de Grado:

\begin{itemize}
    \item Objetivo 3º: Salud y bienestar
\end{itemize}
La mejora en la calidad de las señales obtenidas permite evaluaciones clínicas más precisas, favorece la detección temprana de disfunciones musculares y optimiza los procesos de rehabilitación. El desarrollo de algoritmos más robustos y fiables incide directamente en la calidad de atención sanitaria, apoyando la toma de decisiones de profesionales en el área de la salud.
\begin{itemize}
    \item Objetivo 9º: Industria, innovación e infraestructura.
\end{itemize}
El proyecto impulsa la innovación tecnológica aplicando metodologías de ingeniería como el procesado digital de señales  y el aprendizaje automático a un contexto biomédico. La investigación y comparación de algoritmos de separación refuerza el desarrollo de nuevas herramientas que puede integrarse en dispositivos médicos, prótesis inteligentes o sistemas de monitorización avanzados. Por ello, el trabajo contribuye al fortalecimiento de infraestructuras científicas y tecnológicas orientadas a la mejora de la salud.
\begin{itemize}
    \item Objetivo 10º: Reducción de desigualdades
\end{itemize}
Al avanzar en técnicas que permiten obtener información muscular precisa sin emplear un equipamiento de alto coste y, por ello,  más accesible. Este proyecto favorece que tecnologías basadas en EMG puedan extenderse a entornos clínicos y de investigación con menor disponibilidad de recursos. La estandarización de métodos de separación y su validación permiten su recreación de manera fiable, lo que facilita que estas herramientas se utilicen en una mayor variedad de centros. De este modo, el proyecto promueve un acceso más equitativo a tecnologías de prótesis, contribuyendo a la reducción de desigualdades en el ámbito sanitario.
\clearpage
\chapter{Metodología del trabajo}
En esta sección se expondrá la planificación del trabajo para su adecuada finalización. Se nombrarán los diferentes algoritmos de los que se dispone, pruebas que se realizarán y los procedimientos que sean pertinentes. Tras su explicación en detalle, se ilustrará con un cronograma o un diagrama de Gantt.
\\La estrategia a seguir para este proyecto se basa en los siguientes pasos:
\begin{itemize}
    \item Preprocesamiento (en matlab)
\end{itemize}
\noindent Las señales EMG captadas por los electrodos, sean invasivos o no, deben procesarse adecuadamente antes de comenzar a entrenar y validar los algoritmos de separación de fuentes. Esto se debe a que la señal \textit{cruda} contiene interferencias (dispositivo de alimentación a 50 Hz por ejemplo), ruido u otros factores que puede complicar el entrenamiento del algoritmo. Para solucionar este problema, la mejor solución es aplicar una serie filtros que eliminen frecuencias artificiales en el EMG y ajustar la señal a las necesidades de ICA y derivados. Por ello, se empleará la \href{https://www.mathworks.com/help/signal/ug/introduction-to-filter-designer.htm}{herramienta de diseño de filtros} que \textbf{MATLAB} trae incorporado. Adicionalmente, se realizarán otros cambios en la señal (como normalización o eliminación de valores atípicos \textit{(outliers)}) con el fin de garantizar un procesamiento más eficaz.
\begin{itemize}
    \item Separación de señales (matlab/python)
\end{itemize}
Esta es probablemente la sección del Trabajo Fin de Grado a la que más tiempo y recursos se le dedicará puesto que es la que trata a fondo la complejidad del proyecto. La idea es presentar a nivel teórico y conceptual los diferentes algoritmos de separación de fuentes que se pueden considerar para abordar la temática del trabajo y probarlos en datos reales. Posteriormente, se identificará el método o combinación de métodos que mejor ha satisfecho los objetivos del TFG, justificándose la elección basada en criterios y métricas. \\ 
A continuación se detallan los tres grupos principales de algoritmos que se considerarán en el proyecto.

\begin{itemize}
    \item []
    \begin{itemize}
        \item Separación de fuentes ciega \textit{(Blind Source Separation, BSS)}
        \\ \noindent Esta técnica se basa en una estrategia fundamental de Machine Learning conocida como \textbf{\textit{Independent Component Analysis (ICA)}}, la cuál consiste en encontrar las señales subyacentes independientes a partir de una mezcla observada. Para aplicarse debe asumirse que las fuentes son estadísticamente independientes y que presentan distribuciones NO Gaussianas. Implementaciones famosas de este algoritmo son \textit{FastICA, Jade o Infomax}. S
    \end{itemize}
\end{itemize}

\begin{itemize}
    \item[]
    \begin{itemize}
        \item Separación basada en correlación temporal
        \\ \noindent Este enfoque se centra en la búsqueda de componentes basada en las diferencia en autocorrelación o estructuras temporales. El algoritmo más representativo en esta clasifiación es conocido como \textbf{\textit{SOBI (Second-Order Blind Identification)}}
    \end{itemize}
\end{itemize}
\begin{itemize}
    \item[]
    \begin{itemize}
        \item Separación con referencia o semi-ciega
        \\ \noindent A diferencia de las algoritmos de ICA tradicionales, esta técnica emplea información adicional sobre las señales a descomponer para poder optimizar la separación según las características deseadas. No obstante, todavía no se ha llegado a un consenso generalizado en su desarrollo y no se dispone de funciones o librerías públicas que permitan aplicar este concepto. Por ello, \uline{se abordará principalmente de manera teórica y conceptual}, comentándose implementaciones como \textbf{\textit{Correlation Constrained ICA}}, \textbf{\textit{Kernel Constrained ICA}} y la combinación de ambas, \textbf{\textit{2CFastICA (Kernel \& Correlation Constrained)}}, publicado por \textbf{\textit{\cite{chen_2cfastica_2024}}}
    \end{itemize} 
\end{itemize}
\begin{itemize}
    \item Análisis y validación
\end{itemize}
Una vez separada las señales, es crucial reconocer e identificar cual de los componentes resultantes procede realmente de la señal de los nervios redirigidos frente a otros músculos. Para ello, se debe diseñar un procedimiento capaz de distinguir de manera fiable la fuente reinnervada y emplear métricas, como SNR o correlación, para valorar los resultados.

Una vez mencionado y explicado , se debe organizar. Para ello, la mejor estrategia es idear un cronograma que considere los pasos a seguir a lo largo del Trabajo Fin de Grado y su duración temporal estimada para poder garantizar un desarrollo progresivo y continuado. 

La figura que se muestra a continuación muestra el reparto temporal de las tareas del proyecto. Cabe destacar que esta organización es tentativa y puede sufrir cambios a medida que se avanza en el Trabajo Fin de Grado.

\begin{figure}[H]
    \centering
    \includegraphics[width=\linewidth]{Anexo B/figuras/cronograma.png}
    \caption{Cronograma de las tareas a realizar a lo largo del TFG}
    \label{fig:placeholder}
\end{figure}

\clearpage

\chapter{Recursos a emplear}

Finalmente, en esta sección se presentan de manera detallada los recursos en los que se basará el desarrollo de este Trabajo Fin de Grado. Con esta sección se pretende ofrecer una visión clara y transparente de las herramientas que permitirán implementar, analizar y validar las distintas etapas del proyecto, desde el preprocesamiento de los datos hasta la ejecución de los algoritmos de separación. Se pretende así garantizar la reproducibilidad y coherencia metodológica del trabajo. 


\noindent\includegraphics[height=1.2em]{Anexo B/figuras/python_logo.png}
\hspace{0.5em}%
\noindent\textbf{Python 3.11} \\
\textit{Entorno de desarrollo:}
\begin{itemize}
    \item Visual Studio Code como entorno de programación y depuración.
\end{itemize}
\textit{Librerías a emplear:}
\begin{itemize}
    \item \textbf{scikit-learn}: Para preprocesamiento, normalización, PCA, métricas y evaluación de modelos.
    \item \textbf{MNE-Python}: Para manejo de señales biomédicas, filtrado, visualización y estructuras de datos multicanal.
    \item \textbf{Picard}: Implementación del algoritmo ICA con precondicionamiento y optimización natural, para separación de señales en entornos ruidosos.
\end{itemize}
\includegraphics[height=1.2em]{Anexo B/figuras/Matlab_Logo.png}
\hspace{0.5em}%
\noindent\textbf{MATLAB 2022a} \\
\textit{Entorno de trabajo:}
\begin{itemize}
    \item Editor integrado de MATLAB
\end{itemize}
\textit{Toolboxes y paquetes a emplear:}
\begin{itemize}
    \item \textbf{Signal Processing Toolbox}: Para filtrado digital, análisis espectral y diseño de filtros.
    \item \textbf{Statistics and Machine Learning Toolbox}: Para análisis estadístico, PCA y validación cuantitativa.
    \item \textbf{FastICA package}: Paquete externo para la implementación del algoritmo FastICA como referencia comparativa.
\end{itemize}

\noindent\textbf{Gestión del Proyecto} 
\begin{itemize}
    \item \textbf{GitHub}: Sistema de control de versiones utilizado para el almacenamiento, seguimiento, documentación y evolución del código desarrollado en Python y MATLAB. A continuación se comparte un enlace al repositorio donde se subirán y gestionarám todos los archivos referentes al Trabajo Fin de Grado: \href{https://github.com/acobianiregui/TFG}{acobianiregui/TFG}
    \item \textbf{Zotero}: Gestor bibliográfico empleado para la organización de artículos científicos, normativas, documentación y referencias citadas en la memoria.
\end{itemize}
\noindent\textbf{Redacción del Documento} 
\begin{itemize}
    \item \textbf{LaTeX (Overleaf)}: Plataforma principal para la edición colaborativa y compilación en LaTeX de la memoria del TFG. Incluye gestión de bibliografía mediante \texttt{BibTeX} integrada con Zotero.
\end{itemize}

\nocite{*}
\pagestyle{fancy}
\printbibliography
\end{document}