%preambulo
\documentclass[12pt, a4paper]{report} 

\usepackage[utf8]{inputenc} 
\usepackage[T1]{fontenc} 
\usepackage[spanish]{babel} 
\usepackage{amsmath, amsfonts, amssymb} 
\usepackage{fancyhdr} 
\usepackage{titlesec}
\usepackage{graphicx} 
\usepackage{booktabs} 
\usepackage{float}
\usepackage{caption}
\captionsetup[table]{labelformat=simple} % Aqui para la nomenclatura de las tablas
%Plots / señales... 
\usepackage{pgfplots}
\pgfplotsset{compat=1.18}
\usepackage{subcaption} %Subfiguras

%Margenes
\usepackage[a4paper,
            top=2.54cm,
            bottom=2.54cm,
            left=2.54cm,
            right=2.54cm]{geometry}

%Links 
\usepackage{hyperref} 
\hypersetup{
    colorlinks=false,
    }
\hypersetup{
  colorlinks=true,
  linkcolor=black,      % enlaces internos (como footnotes)
  citecolor=black,      % enlaces a bibliografía
  urlcolor=blue,        % enlaces web
  pdfborder={0 0 0},     % <-- elimina los recuadros de 
  pdftitle={TFG ACI},
  pdfpagemode=FullScreen
}
%Codigo
\usepackage{listings} 
\usepackage{minted}
\usepackage{xcolor}
\usepackage[final]{pdfpages}
\usepackage{lipsum}
\usepackage{changepage}
\usepackage{xtab}
\usepackage[most]{tcolorbox}
\usepackage[shortlabels]{enumitem}
%Referencias 
\usepackage{csquotes}

\usepackage[
    backend=biber,
    sorting=ynt,
    style=apa
]{biblatex}
\addbibresource{referencias2.bib}
\fancyhf{} % Limpia encabezado
% Capitulos       
\titleformat{\chapter}[display]
  {\normalfont\huge\bfseries}
  {\chaptername\ \thechapter}{10pt}{\Huge}
\titlespacing*{\chapter}{0pt}{-60pt}{30pt}

\begin{document}

\chapter*{Boceto Anexo B}
\section*{1. Introducción}
Esta sección detallará la contextualización del trabajo en el ámbito de la ingeniería de telecomunicaciones y sus posibles aplicaciones en otros campos, como la ingeniería biómedica.
\section*{2. Estado de la cuestión}
Actualmente se dispone de una gran variedad de estudios que investigan los algoritmos de separación de señales y sus aplicaciones en ámbitos como imágenes, sonido e incluso señales EEG. No obstante, la implementación de estas técnicas en señales electromiográficas permanece relativamente poco explorada, a pesar de su potencial para ofrecer información valiosa sobre músculos reinervados.  (A expandir)
\section*{3. Motivación}
\section*{4. Objetivos del proyecto}
A continuación se detallan los objetivos del Trabajao Fin de Grado, explicándose más a fondo en la versión final del Anexo B.  
\begin{itemize}
    \item Analizar y tratar la señal del músculo reinnervado.
    \item Separación de la señal EMG, resultando en la señal del nervio y el músculo auxiliar
    \item Discusión de varios algoritmos, exponiendo sus pros y contras, y finalmente seleccionando uno para 
    \item Explicar el uso y comportamiento del algortimo para el público menos técnico (ejemplo: manual de uso para personal sanitario)
    
\end{itemize}

\section*{5. Alineación con ODS}
\href{Link}{https://sdgs.un.org/goals}
\begin{itemize}
    \item Objetivo 3º: Salud y bienestar
    \item Objetivo 9º: Industria, innovación e infraestructura.
\item Objetivo 16º: Reducción de desigualdades
\end{itemize}
\section*{6. Metodología del trabajo}
En esta sección se expondrá la planificación del trabajo para su adecuada finalización. Se nombrarán los diferentes algoritmos de los que se dispone, pruebas que se realizarán y los procedimientos que sean pertinentes. Tras su explicación en detalle, se ilustrará con un cronograma o un diagrama de Gantt.
\section*{7. Recursos empleados}
Los recursos a emplear en este trabajo incluyen otros algoritmos existentes, así como lenguajes y herramientas de programación. Entre ellos se encuentran python, matlab, Github, Zotero...
\\Los algoritmos más relevantes para este trabajo son los mencionados a continuación:
\begin{itemize}
    \item ICA (Separación a ciegas pura)
    \item Constrained ICA (Referencia conocida)
    \begin{itemize}
        \item Kernel Constrained
        \item Correlation Constrained
    \end{itemize} 
    \item Non-linear 
    \begin{itemize}
        \item SOBI
        \item Non-linear ICA
        \item Convolutive ICA
    \end{itemize}
    
\end{itemize}

//

A continuación se adjuntan referencias que posiblemente se emplearán a lo largo del proyecto:



\end{document}